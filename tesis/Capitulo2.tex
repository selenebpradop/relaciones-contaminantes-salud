\chapter{Antecedentes}

%En algunos trabajos se han utilizado modelos de series de tiempo y ...
Existen factores ambientales que afectan la salud de una comunidad como: el abastecimiento de agua potable y el saneamiento, la vivienda y el hábitat, la alimentación, la contaminación ambiental, el empleo de productos químicos y los riesgos ocupacionales \citep{r2}. 

Contaminación del aire es un término usado para describir la presencia de uno o más contaminantes en la atmósfera, cuyas cantidades y características pueden resultar perjudiciales o interferir con la salud, el bienestar u otros procesos ambientales naturales \citep{r3}.

En el presente capítulo se presentan los fundamentos y definiciones de los conceptos más relevantes para el tema de estudio abordado.
%\section{Antecedentes históricos}

\section{Monitoreo de calidad del aire}
Existen diversos estudios que muestran que existen potenciales efectos a la salud cuando en el aire están presentes contaminantes en forma de partículas, gases o agentes biológicos.
%En los últimos años ha habido un desarrollo considerable de la tecnología para el control de la calidad del aire, esto como resultado de una mayor conciencia, tanto de parte de los gobiernos como de los ciudadanos, sobre la importancia de mantener el aire lo más apto posible para la vida humana.

\citet{r4} mencionan que desde inicios de 1950 se observa una preocupación por los contaminantes del aire  en los países de América Latina y el Caribe. Las universidades y dependencias de los ministerios de salud fueron los organismos que realizaron las primeras mediciones de contaminación en el aire.

En 1965, el Consejo Directivo de la Organización Panamericana de la Salud (OPS) recomendó el establecer programas de investigación de la contaminación del agua y del aire, con el objetivo de colaborar en el desarrollo de políticas adecuadas de control \citep{r5}.

Mediante el Centro Panamericano de Ingeniería Sanitaria y Ciencias del Ambiente (CEPIS), la OPS acordó establecer una red de estaciones de muestreo de la contaminación del aire.
En junio de 1967 La Red Panamericana de Muestreo Normalizado de la Contaminación del Aire (REDPANAIRE) inició sus operaciones recolectando muestras mensuales de polvo sedimentable (PS) y muestras diarias de partículas totales en suspensión (PTS) y de SO2. La REDPANAIRE comenzó con ocho estaciones y a fines de 1973 tenía un total de 88 estaciones distribuidas en 26 ciudades de 14 países \citep{r5}.

Para diciembre de 1973 se habían recolectado más de 350,000 datos sobre la calidad del aire, en los que se observa que algunas ciudades mostraban una tendencia al incremento de los niveles de contaminación \citep{r5}.

En 1980 la REDPANAIRE desapareció y pasó a formar parte del Programa Global de Monitoreo de la Calidad del Aire, iniciado en 1976 por la OMS y el Programa de las Naciones Unidas para el Medio Ambiente (PNUMA), como parte de un sistema global de monitoreo ambiental llamado GEMS por sus siglas en inglés \emph{Global Environmental Monitoring System}.

En la década de 1990, la OMS organizó, con carácter global, el Sistema de Información para el Control de la Calidad del Aire llamado AMIS por sus siglas en inglés \emph{Air Management Information System}. Entre las actividades más destacadas de AMIS se incluye el coordinar las bases de datos sobre temas relacionados con la calidad del aire.

En Nuevo León, México, las operaciones de la Red Automática de Monitoreo Atmosférico iniciaron en 1993. Dicha red en sus inicios contaba con cinco estaciones fijas de monitoreo continuo de monóxido de carbono (CO), dióxido de azufre (SO$_2$), óxidos de nitrógeno (NO$_x$), ozono y PM10 \citep{r4}. Como se muestra en la figura \ref{estaciones}, actualmente se cuenta con nueve estaciones fijas.

\section{Series de tiempo}
\citet{r4} mencionan que las relaciones entre niveles de concentraciones de contaminantes del aire y los efectos sobre la salud generalmente son obtenidas de estudios epidemiológicos de series de tiempo. Uno de los diseños epidemiológicos más utilizados son los estudios de series temporales. Con esos diseños se analizan las variaciones en el tiempo de la exposición al contaminante y el indicador de salud estudiado en una población \citep{r1}.

Las series de tiempo se pueden definir como un conjunto de observaciones {{ot}} tomadas en un tiempo \emph{t} determinado. Los estudios de series de tiempo relacionan estadísticamente los cambios temporales en la repercusión de cambios en la concentración de un contaminante en la población \citep{r8}.

Para mostrar datos en una serie de tiempo, especialmente en el área médica, estos suelen agruparse en \emph{semanas epidemiológicas}\footnote{Una semana epidemiológica es un estándar de medición temporal que se utiliza para comparar datos en ventanas de tiempo definidas. La primera semana epidemiológica del año termina el primer sábado de enero de cada año \citep{r7}.}. 

\section{Clasificación de enfermedades}
Existe un instrumento estadíıstico y sanitario para identificar enfermedades llamado Clasificación Internacional de Enfermedades (CIE), cuya finalidad es entender las causas de morbilidad y mortalidad de la población y así mejorar la calidad de vida de la misma. Es en base a un criterio epidemiológico y sanitario establecido por Farr a finales del siglo XIX que esta clasificación agrupa enfermedades en epidémicas, generales, locales ordenadas por origen geográfico, trastornos del desarrollo y lesiones \citep{r9}. Para lograr distinguirlas se emplea un código alfanumérico que consiste de una letra en la primera posición, seguida de dos dígitos, un punto decimal y un último dígito. El rango de valores va de A00.0 a Z99.9.

\section{Regresión lineal}
La tendencia \emph{$w_0$} de una serie de tiempo puede ser obtenida a partir de una regresión lineal de la misma \citep{r10}. Una regresión lineal es una metodología inferencial supervisada que busca predecir valores \emph{y} dado un vector de variables de entrada \emph{t} por medio del ajuste de coeficientes \emph{w} de la función lineal 
\begin{equation}
    \hat{y}(t,w) = w_0 + w_1x_1 + ... + w_tx_t.
    \label{eq1}
\end{equation}

\section{Regresión lineal múltiple}
Un modelo de regresión múltiple es un modelo complemento de la regresión lineal simple, el cual tiene dos o más variables independientes \emph{k} que pueden influir en una variable dependiente \emph{y}. \citet{r11} expresan la regresión múltiple mediante la siguiente ecuación:
\begin{equation}
    y = \beta_0 + \beta_1x_1 + ... + \beta_kx_k + \varepsilon.
    \label{eq2}
\end{equation}

%\section{Modelos ARIMA}
%Los modelos autorregresivos integrados de media móvil o ARIMA, por su abreviatura en inglés abarcan un catalogo de aproximaciones para el estudio de series de tiempo. 

%Los modelos ARIMA utilizan variaciones y regresiones de datos estadísticos con el fin de encontrar patrones para una predicción hacia el futuro. 

