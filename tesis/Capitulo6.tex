\chapter{Conclusiones}

El presente capítulo describe la tesis a partir de la manera que cumple los objetivos generales y específicos para determinar si la hipótesis se comprueba, trata también del porqué se realizó la tesis.

En el presente proyecto se generaron visualizaciones para el estudio de las relaciones entre determinados contaminantes y determinadas CIE. Además, se generaron modelos de regresión lineal y modelos de regresión lineal múltiple con diferentes contaminantes y diferentes CIE para poder realizar una comparación entre los modelos generados.

En los experimentos se encontró un coeficiente de correlación de Pearson diferente a 0 entre -1 y 1, por lo tanto se pudieron generar los modelos de regresión lineal. Se encontró que al estudiar las CIE de manera agrupada en un modelo de regresión múltiple se obtiene una mejor explicación de la varianza de los niveles de los contaminantes PM10 y PM2.5 frente al modelo de regresión lineal simple. Cinco de los veinticuatro valores de error RMSE reportados son menores a 0.20, lo cual indica que la mayoría de los valores predichos por los modelos se alejaron más de 0.20 unidades de los valores reales, por lo tanto dichos valores de error son mayores a los deseados.

\section{Contribuciones}
Primeramente se encontró que la CIE que reporta mayor número de egresos en Nuevo León, México en los años 2017 y 2018 es la CIE O809. En las series de tiempo se observa que la cantidad de egresos de la mayoría de las CIE estudiadas presentar una línea de evolución similar al contaminante PM10.

Además, se encontró una correlación lineal entre los contaminantes estudiados y las CIE estudiadas, por lo cual el presente proyecto motiva a seguir realizando labores en torno a la investigación de la dependencia lineal que se presenta entre los contaminantes y las CIE.

Finalmente, se observó que para el estudio de las relaciones entre los contaminantes y las CIE se obtuvieron mejores resultados con los modelos de regresión lineal frente a los modelos de regresión lineal simple, lo cual indica que se puede obtener información relevante si se emplean modelos de regresión lineal múltiple para realizar investigaciones que estudien las relaciones entre los niveles de determinados contaminantes y el número de egresos por determinadas CIE.

\clearpage
\section{Trabajo a futuro}
El presente trabajo brinda algunos aspectos a considerar para realizar un trabajo a futuro, los cuales son: la recolección de más datos de los niveles de contaminantes y de egresos para el estudio de años más recientes, realizar un estudio de las relaciones de los niveles de determinados contaminantes y la cantidad de egresos por CIE empleando modelos de regresión lineal múltiple, la creación de un mapa interactivo donde se pueda observar por región los niveles de los contaminantes y la cantidad de egresos, y la generación de una página web que funcione en conjunto con el desarrollo elaborado para que al ingresar el archivo con los datos se generen las visualizaciones y modelos.

Las oportunidades de mejora detectada para el presente proyecto son: comparar los resultados obtenidos de los modelos regresión lineal múltiple con el mismo modelo pero incrementando el número de variables independientes (CIE), elaborar modelos de regresión no lineal para comparar sus resultados con los resultados de los modelos generados, y utilizar un conjunto de datos más consistente ya que eso favorece a que se tengan resultados más fiables y precisos.


% PENDIENTES:
% RESUMEN EN INGLES