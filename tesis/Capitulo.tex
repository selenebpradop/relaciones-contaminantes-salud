\chapter{Introducción}
Analizar y cuantificar...
\clearpage

\section{Motivación}
Pese a que ya...

\section{Hipótesis}
Se sabe que...

\section{Objetivos}
En esta sección se establece el objetivo general y los objetivos específicos sobre los que se enfoca la tesis...

\subsection{Objetivo general}
El objetivo de realizar...

\subsection{Objetivos específicos}
\begin{itemize}
\item Generar...
\end{itemize}

\section{Estructura}
El contenido de la investigación se divide en...


\chapter{Antecedentes}
Hoy en día existen diversas tecnologías que...

\section{Antecedentes históricos}
El procesamiento de...

\section{Descriptores de características globales}
La idea de que... 

\subsection{Color}
La característica de clasificación de color...

\subsection{Forma}
La característica de forma...

\subsection{Textura} 
Esta característica...

\section{Descriptores de características locales}
Las características...

\section{Uso de los descriptores}
Existen...


\chapter{Estado del arte}
En este capítulo se explica...

\section{Investigaciones relacionadas}
Existen algunos trabajos que...

\section{Comparación de trabajos}
La mayoría de los trabajos citados...

\subsection{Comparaciones}
En el cuadro...

\subsection{Áreas de oportunidad}
En el cuadro...


\chapter{Solución propuesta}
Habiendo conocido las características que mejor describen a los atributos del presente trabajo, se puede decir que la base del método propuesto se puede desarrollar...

\section{Fase de recolección de muestras}
La primera fase en el desarrollo de la solución propuesta sería...

\section{Muestras recolectadas}
Inicialmente...

\subsection{Análisis de muestras}
Como se mencionó...

\subsection{Información no útil}
En cada una de las muestras recolectadas...

\section{Fase de procesamiento de muestras}
Una vez recolectadas...

\subsection{Recortando muestras}
Primeramente hay reconocer...


\chapter{Desarrollo de la solución}
Recapitulando las fases anteriores, se conoce que...

\section{Fase de entrenamiento}
Originalmente...

\section{Fase de detección}
Esta fase es la más importante de todas debido a que se utiliza el modelo generado a partir de la fase de entrenamiento. En esta fase se utilizan... 

\section{Fase de combinación}
La fase de combinación...

\chapter{Experimentos}
Después de...

\section{Diseño experimental}
Hola...

\section{Resultados}
Establecidos los experimentos que se van a realizar, se reporta los resultados obtenidos...

\section{Discusión}
Todos los experimentos son ejecutados en una laptop con las especificaciones del cuadro \ref{tab:Especificaciones técnicas del PC}.

\begin{table}[H]
	{\centering
		\caption{Especificaciones técnicas del equipo de cómputo}
		\begin{tabular}{|c|c|c|}
			\hline
			Sistema Operativo & Windows 10 64 bits\\
			\hline
			Procesador & Intel Core i5-7300HQ\\
			\hline
			RAM & 8 GB RAM DDR4 2133 MHz\\
			\hline
		\end{tabular}

	\label{tab:Especificaciones técnicas del PC}
	}
\end{table}


\chapter{Conclusiones}
Este capítulo describe la tesis a partir de la manera que cumple los objetivos generales y específicos para determinar si la hipótesis se comprueba, trata también del porque se realizó la tesis...

\clearpage

\section{Contribuciones}
La solución propuesta surgió a partir de...

\section{Trabajo a futuro}
La solución propuesta en la tesis...