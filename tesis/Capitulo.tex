\chapter{Introducción}
El crear modelos para la visualización de datos ayuda a observar con mayor claridad los datos para encontrar relaciones entre ellos.

El aprendizaje máquina es un área dentro de la ciencia de datos que ayuda a crear dichos modelos para tener una más eficiente visualización de los datos cuando se trabaja con una gran cantidad de datos. El área de la ciencia de datos es muy útil ya que permite trabajar con grandes cantidades de datos aminorando la cantidad de tiempo empleado en la creación de gráficos que permitan visualizar los datos.
\clearpage

\section{Motivación}
Existen investigaciones que ya han estudiado las relaciones entre contaminantes del aire y salud pública, sin embargo, con el presente trabajo se busca aportar a la creación de nuevas herramientas que permitan observar y estudiar dichas relaciones. El poder visualizar dichas relaciones puede ayudar a tomar medidas adecuadas que permitan aminorar los efectos negativos de los contaminantes del aire en las personas.

\section{Hipótesis}
Se plantea que con modelos de regresión se pueden obtener gráficos donde se pueden observar las relaciones entre el número de ingresos hospitalarios y los niveles de contaminantes del aire.

\section{Objetivos}
En esta sección se establece el objetivo general y los objetivos específicos sobre los que se enfoca la tesis...

\subsection{Objetivo general}
El objetivo de realizar...

\subsection{Objetivos específicos}
\begin{itemize}
\item Generar...
\end{itemize}

\section{Estructura}
El contenido de la investigación se divide en...


\chapter{Antecedentes}
Hoy en día existen diversas tecnologías que...

\section{Antecedentes históricos}
El procesamiento de...


\chapter{Estado del arte}
En este capítulo se explica...

\section{Investigaciones relacionadas}
Existen algunos trabajos que...

\section{Comparación de trabajos}
La mayoría de los trabajos citados...

\subsection{Comparaciones}
En el cuadro...

\subsection{Áreas de oportunidad}
En el cuadro...


\chapter{Solución propuesta}
Habiendo conocido las características que mejor describen a los atributos del presente trabajo, se puede decir que la base del método propuesto se puede desarrollar...

\section{Datos recolectados}
Inicialmente...


\chapter{Desarrollo de la solución}
Recapitulando las fases anteriores, se conoce que...


\chapter{Experimentos}
Después de...

\section{Diseño experimental}
Hola...

\section{Resultados}
Establecidos los experimentos que se van a realizar, se reporta los resultados obtenidos...

\section{Discusión}
Todos los experimentos son ejecutados en una laptop con las especificaciones del cuadro \ref{tab:Especificaciones técnicas del PC}.

\begin{table}[H]
	{\centering
		\caption{Especificaciones técnicas del equipo de cómputo}
		\begin{tabular}{|c|c|c|}
			\hline
			Sistema Operativo & Windows 10 64 bits\\
			\hline
			Procesador & Intel Core i5-7300HQ\\
			\hline
			RAM & 8 GB RAM DDR4 2133 MHz\\
			\hline
		\end{tabular}

	\label{tab:Especificaciones técnicas del PC}
	}
\end{table}


\chapter{Conclusiones}
Este capítulo describe la tesis a partir de la manera que cumple los objetivos generales y específicos para determinar si la hipótesis se comprueba, trata también del porque se realizó la tesis...

\clearpage

\section{Contribuciones}
La solución propuesta surgió a partir de...

\section{Trabajo a futuro}
La solución propuesta en la tesis...