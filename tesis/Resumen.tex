%Resumen

\chapter{Resumen}
\markboth{Resumen}{}

{\setlength{\leftskip}{10mm}
\setlength{\parindent}{-10mm}

\autor.

Candidato para obtener el grado de \grado\orientacion.

\uanl.

\fime.

Título del estudio: \textsc{\titulo}.

\noindent Número de páginas: \pageref*{lastpage}.}

%%% Comienza a llenar aquí
\paragraph{Objetivos y método de estudio:}
El objetivo de la investigación es generar modelos que permitan visualizar relaciones entre contaminantes atmosféricos y salud pública. Los modelos generados se utilizan en conjunto con datos obtenidos de la Secretaría de Salud del Gobierno de México y registros de los niveles de los contaminantes presentes en el área metropolitana de Monterrey. 

El tener un modelo que permita visualizar relaciones entre contaminantes atmosféricos y salud pública que sea utilizado con datos confiables y verídicos pueden ayudar a visualizar el impacto que tiene el aumento del nivel de contaminantes atmosféricos.
\paragraph{Contribuciones y conclusiones:}
Durante la investigación se exploraron diversas maneras de visualizar la información, además de generar modelos que permiten analizar las relaciones existentes entre los niveles de determinados contaminantes y determinadas CIE. Los tipos de visualizaciones generadas son series de tiempo y gráficos de radar y, los modelos generados son modelos de regresión lineal y modelos de regresión lineal múltiple. 

\bigskip\noindent\begin{tabular}{lc}
\vspace*{-2mm}\hspace*{-2mm}Firma de la asesora: & \\
\cline{2-2} & \hspace*{1em}\asesor\hspace*{1em}
\end{tabular}

\bigskip\noindent\begin{tabular}{lc}
\vspace*{-2mm}\hspace*{-2mm}Firma de la coasesora: & \\
\cline{2-2} & \hspace*{1em}\revisorA\hspace*{1em}
\end{tabular}

